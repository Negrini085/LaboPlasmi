\chapter{Procedure di Laboratorio}

In questo capitolo sono esposte le principali tecniche e procedure utilizzate in laboratorio, con un'attenzione particolare
al setup sperimentale e agli strumenti di misura presenti. 

\section{Utilizzo della strumentazione}

In primo luogo è necessario produrre il campo magnetico che consente di intrappolare le particelle cariche: il campo \textit{solenoidale} 
è generato da una bobina che, se percorsa da $300\,A$ di corrente, determina la presenza di un campo pari a $0.1\,T$. Noi 
lavoriamo solitamente con $I\,=\,390\,A$: data la linearità della relazione possiamo facimente ricavarci $B$.

Oltre alla bobina principale, sono presenti quattro bobine di correzione per limitare il più possibile il disallineamento e le
irregolarità del campo: notiamo infatti che se le scolleghiamo il plasma fuoriesce dalla trappola con una differente conformazione.
I valori nominali delle quattro correnti sono: $0.48\,A,\,0.18\,A,\,1.06\,A,\,3.00\,A$.

\subsection{Temporizzazione}

Il setup sperimentale presente in laboratorio è molto complesso: come si riesce a gestire il macchinario e a produrre le 
condizioni sperimentali ricercate? Un generatore ad 8 canali consente di temporizzare i potenziali nella trappola, specificandone 
l'ampiezza e l'intervallo di tempo per i quali essi devono agire. Il programma che gestisce il generatore accetta come input un
file di testo contenente le istruzioni necessarie: le seguenti righe di codice riportano un esempio di file di istruzioni
funzionante ed utilizzato in laboratorio. 
\begin{verbatim}
    *Clock	1ms							
    *Name	    ENDC	ENDP	ZERO	CH4	    CAM	    FD	    SW	    OSC
    *Channels	1	    2	    3	    4	    5	    6	    7	    8
    *Amplitude	100	    100	    100	    100	    100 	100	    100 	100
    10000	    0	    0	    0	    0	    0	    0	    3	    0
    2000	    0	    0	    0	    0	    0	    3	    0	    0
    500	        5	    0	    0	    0	    0	    0	    0	    1
    500	        5	    5	    0	    0	    0	    0	    0	    0
\end{verbatim}
La prima riga consente di scegliere la scala di tempo con la quale di vogliono specificare le istruzioni: le possibili scelte
sono $1\,ms$ oppure $1\,\mu s$ (da indicare però come $1MHz$). La seconda e la terza riga consentono di associare delle
etichette ai vari canali presenti: solitamente i primi due sono riservati ai potenziali di confinamento, l'ultimo all'
oscilloscopio e quello indicato dal nome FD al feedback damping. Il canale 7 è riservato allo switch per generare il plasma, 
mentre il canale 5 è quello del trigger della camera CCD.
I primi due canali non riportano i valori del potenziale di confinamento, ma sono degli interrutori modulari per l'attivazione
del potenziale confinante: i possibili valori variano fra 0 (potenziale più negativo) e 5 (potenziale più positivo). Notiamo 
che nella penultima riga il canale ENDC ha come valore 5: questo accade perchè vogliamo rimuovere la barriera di potenziale
presente nell'estremità dove è posto il collettore per valutare la quantità di carica del plasma prodotto. La riga precedente
abilita l'utilizzo del feedback damping: questo consente di riportare il plasma prodotto (solitamente fuori asse) verso il 
centro della trappola.

Le istruzioni presenti sono \textbf{sequenziali}: ogni singola riga non può avere una durata superiore alle 30000 unità
di tempo selezionate e su ogni colonna deve essere riportata l'informazione relativa allo specifico canale.\\

Una volta prodotto il file di controllo, lo si deve fornire come input del programma \textbf{Waveform Generator}: per caricarlo
si deve seguire il path \textit{file/import} e svolgere le istruzioni che compaiono a video. Selezionando \textit{output} e 
\textit{configure channel} è possibile verificare la corretta corrispondenza dei canali: una volta aver effettuato questa 
configurazione è necessario selezionale l'opzione \textit{send all}.

\subsection{Oscilloscopio}

L'oscilloscopio che viene utilizzato in laboratorio presenta 4 canali:
\begin{enumerate}[itemsep=0pt]
    \item segnale proveniente dal \textbf{collettore} \\
    \item amplificatore in trans-impedenza, con segnale proveniente da \textbf{$S_{2b}$}    \\
    \item trigger dell'oscilloscopio, con soglia pari a $500\,mV$ \\
    \item libero  
\end{enumerate}
Nel momento in cui vengono salvate le misurazioni, il file che viene prodotto contiene solamente i valori numerici 
delle quantità che sono in quel momento visibili sull'oscilloscopio: lavorando con il secondo canale possiamo apprezzare
l'efficacia del \textit{feedback damping}, che smorza esponenzialmente il segnale legato all'offset del plasma prodotto
in $6/7\,ms$. L'oscilloscopio presenta 12 bit verticali, ossia $2^{12}$ divisioni: la  risoluzione in ordinata può essere
molto elevata. Lo strumento presente in laboratorio può produrre degli errori in presa dati: quando il segnale preso in analisi
è fuori scala potrebbe accadere che a volte produca dei valori infiniti (nel file di output sono presenti i simboli $\infty$).
Per evitare questa problematica è necessario lavorare a scale differenti, che consentano di contenere nella schermata tutto
il segnale che si vuole analizzare.

\subsection{Fotocamera}

La fotocamera uzilizzata scatta delle foto di dimensioni \verb|2448 x 2048|: tali valori sono eccessivi, per questo motivo è
consigliabile effettuare un binning per dimezzare sia il numero di righe che il numero di colonne. Il tempo di esposizione
con cui vengono stampate le foto dipende dallo scopo con cui viene scattata l'immagine: $t_{esp}\,=\,2\,ms$ se vogliamo 
investigare del plasma, oppure $t\,=\,2\,s$ se vogliamo valutare le dimensioni, in metri, del singolo pixel.

\section{Esperienza 0}

L'esperienza introduttiva in laboratorio consiste nello studio di caratteristiche fondamentali del plasma come per esempio la 
carica totale: una frazione importante dell'esperimento è dedicata alla calibrazione dell'apparato e allo studio delle sistematiche
dovute per esempio alla presenza di rumore.
