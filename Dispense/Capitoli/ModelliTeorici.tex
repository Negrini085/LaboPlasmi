\chapter{Modelli teorici}

In questo capitolo investighiamo gli approcci teorici che consentono di trattare un plasma non collisionale e non neutro: ciò
vuol dire che il sistema presenta una carica netta e che le proprietà sono studiate per intervalli di tempo piccoli rispetto
alle tempistiche medie fra collisioni. Per descrivere un tale plasma possonon essere utilizzati due formalismi:
\begin{itemize}
    \item \textit{trattazione di fluido macroscopico}, basata sulle equazioni di Maxwell e sull'equazione dei momenti
    \item \textit{trattazione cinetica}, basata sulle equazioni di Vlasov-Maxwell
\end{itemize}
Nella prima delle due descrizioni vengono presi in considerazione osservabili macroscopici quali $n\left(\vec{x},\,t\right)$, 
$V\left(\vec{x},\,t\right)$ e $P\left(\vec{x},\,t\right)$: tali quantità evolvono in dipendenza dei campi elettrici e magnetici
presenti, che possono essere determinati mediante le equazioni di Maxwell. Se il plasma in analisi è freddo è possibile trascurare
le variazioni di pressione, ponendo a zero la varianza del tendore delle pressioni. Il vantaggio di un tale approccio è la sua 
elevata semplicità: in ambito fluido non è però possibile trattare instabilità collettive e fenomeni caratteristici di un plasma, 
come il Landau damping.

Per includere la fenomenologia termica nell'analisi del plasma è necessario utilizzare un approccio \textit{cinetico}: in questo
caso l'attenzione è rivolta alla distribuzione $f\left(\vec{x},\,\vec{p},\,t\right)$, che descrive la densità di probabilità di 
avere il sistema nel punto $\left(\vec{x},\,\vec{p}\right)$ dello spazio delle fasi al tempo $t$. Per come è definita la
distribuzione $f$, abbiamo che
$$f\left(\vec{x},\,\vec{p},\,t\right)d^3xd^3p\,=\,\text{numero medio di particelle nell'intorno di } \left(\vec{x},\,\vec{p}\right)$$
Questo approccio consente di studiare un'ampia classe di fenomeni collettivi che dipendono dalla struttura della distribuzione
all'equilibrio nello spazio delle fasi.

\section{Trattazione cinetica}

Consideriamo un plasma non neutro di elettroni caratterizzati da carica $e$ e massa $m$: per scale di tempo corte rispetto
al tempo di collisione la funzione di distribuzione di singola particella $f\left(\vec{x},\,\vec{p},\,t\right)$ evolve secondo
l'equazione di Vlasov, che descrive l'evoluzione incomprimibile secondo il teorema di Liouville nello spazio delle fasi
6-dimensionale $\left(\vec{x},\,\vec{p}\right)$.
\begin{equation}
    \left\{\frac{\partial}{\partial t}\,+\,\vec{v}\cdot\frac{\partial}{\partial \vec{x}}\,+\,e\left(\vec{E}\,+\,\frac{\vec{v}\times\vec{B}}{c}\right)\cdot\frac{\partial}{\partial \vec{p}}\right\}f\left(\vec{x},\,\vec{p},\,t\right)\,=\,0
    \label{equation: VlasovEq}
\end{equation}
Notiamo che è presente un termine in cui figurano sia il campo elettrico che il campo magnetico: possiamo determinarli utilizzando le
equazioni di Maxwell
\begin{align}
    &\nabla \times \vec{E} = -\frac{1}{c}\left[c\right]\frac{\partial \vec{B}}{\partial t} \\
    &\nabla \times \vec{B} = \frac{4\pi}{c}\left[\frac{\mu_0 c}{4\pi}\right]e\int{d^3p\vec{v}f(\vec{x},\,\vec{p},\,t)} + \frac{4\pi}{c}\left[\frac{\mu_0 c}{4\pi}\right]\vec{J}_{ext} + \frac{1}{c}\left[\frac{1}{c}\right] + \frac{\partial \vec{E}}{\partial t} \\
    &\nabla \cdot \vec{E} = 4\pi\left[\frac{1}{4\pi\varepsilon_0}\right]e\int{d^3p f\left(\vec{x},\,\vec{p},\,t\right)} + 4\pi\left[\frac{1}{4\pi\varepsilon_0}\right]\rho_{ext} \\
    &\nabla \cdot \vec{B} = 0
\end{align}
Le equazioni di Vlasov-Maxwell sono altamente non lineari, in quanto $f\left(\vec{x},\,\vec{p},\,t\right)$ è modificata dai campi auto-indotti
dal plasma, che a loro volta evolvono quando la funzione di distribuzione cambia. In condizioni di stato quasi-stazionario è possibile
porre a zero tutte le derivate parziali rispetto al tempo in modo da determinare le soluzioni stazionarie $f^0\left(\vec{x},\,\vec{p}\right)$,
$\vec{E}^0\left(\vec{x}\right)$ e $\vec{B}^0\left(\vec{x}\right)$: lavorando con piccole perturbazioni è possibile valutare la stabilità
degli equilibri così individuati.

\section{Trattazione fluida}

In alcune circostanze il comportamento globale del plasma può essere descritto utilizzando un'approccio fluido-dinamico: 
siamo interessati alla densità del sistema $n\left(\vec{x},\,t\right)$, alla velocità media $\vec{V}\left(\vec{x},\,t\right)$,
al momento medio $\vec{P}\left(\vec{x},\,t\right)$ ed al tensore delle pressioni $\mathbf{P}\left(\vec{x},\,t\right)$. Per 
passare dalla trattazione cinetica a quella fluida introduciamo i primi momenti della distribuzione $f\left(\vec{x},\,\vec{p},\,t\right)$ 
integrando nei momenti, in modo tale da perdere l'informazione cinetica e mantenere solo quella spaziale. Possiamo quindi 
riconoscere:
\begin{equation}
    n\left(\vec{x},\,t\right)\,=\,\int d^3p \,f\left(\vec{x},\,\vec{p},\,t\right)
    \label{equation: numb_density}
\end{equation}
\begin{equation}
    n\left(\vec{x},\,t\right)\vec{V}\left(\vec{x},\,t\right)\,=\,\int d^3p\,\vec{v} f\left(\vec{x},\,\vec{p},\,t\right)
    \label{equation: mean_velocity}
\end{equation}
\begin{equation}
    n\left(\vec{x},\,t\right)\vec{P}\left(\vec{x},\,t\right)\,=\,\int d^3p\,\vec{p} f\left(\vec{x},\,\vec{p},\,t\right)
    \label{equation: p_tens}
\end{equation}
\begin{equation}
    \mathbf{P}\left(\vec{x},\,t\right)\,=\,\int d^3p\left[\vec{p}\,-\,\vec{P}\left(\vec{x},\,t\right)\right]\left[\vec{v}\,-\,\vec{V}\left(\vec{x},\,t\right)\right] f\left(\vec{x},\,\vec{p},\,t\right)
    \label{equation: p_tens1}
\end{equation}
Possiamo ora ottenere le equazioni fluide andando ad integrare nei momenti: la semplice integrazione in $d^3p$ restituisce 
l'equazione di continuità, mentre lavorando con $\vec{p}d^3p$ è possibile ottenere l'equazione che descrive l'equilibrio delle
forze:
\begin{equation}
    \frac{\partial n}{\partial t}\,+\,\nabla \cdot \left(n\vec{x}\right)
    \label{equation: continuity_eq}
\end{equation}
\begin{equation}
    n\left(\frac{\partial}{\partial t}\,+\,\vec{V}\cdot \nabla\right)\vec{P}\,+\,\nabla \cdot \mathbf{P}\,=\,nq\left(\vec{E}\,+\,\frac{1}{c}\left[c\right]\vec{v} \times \vec{b}\right)
    \label{equation: force_balance}
\end{equation}
Abbiamo trovato un sistema di equazioni differenziali in cui nell'equazione che descrive l'evoluzione del momento di ordine 0 
compare il momento di primo ordine ed in generale nell'equazione riguardante il k-esimo momento è presente il k+1-esimo: per 
chiudere il sistema supponiamo di trattare un plasma freddo, ossia caratterizzato da velocità termica molto inferiore alla 
velocità fluida in gioco. Così facendo possiamo trascurare il termine di ordine superiore, ossia la divergenza del tensore delle 
pressioni, di fatto trascurando l'agitazione termica rispetto ai moti del fluido.
