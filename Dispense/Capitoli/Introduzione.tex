\chapter{Introduzione}

Nel corso di Laboratorio di Fisica dei Plasmi siamo interessati a plasmi non neutri, ossia dei sistemi a molti corpi
costituiti da particelle cariche, che presentano una carica netta globale. Sistemi di questo genere sono caratterizzati
dalla presenza di forti campi elettrici legati alla natura non neutra dell'insieme di particelle considerato: tali campi
possono avere grande influenza sul comportamento del plasma e sulle proprietà del regime di stabilità che si può instaurare.
I plasmi generati in laboratorio sono plasmi di elettroni: essi vengono confinati utilizzando una trappola di Malmberg-Penning.

\section{Struttura della dispensa}

La dispensa si articola nei seguenti capitoli:
\begin{enumerate}
    \item
  \end{enumerate}
