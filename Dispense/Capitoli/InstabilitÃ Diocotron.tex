\chapter{Instabilità di diocotron}

Nel capitolo precedente, la velocità angolare era indipendente dalla posizione radiale (ossia $\partial \omega/\partial r\,=\,0$): 
in questo capitolo non è più presente questa restrizione poichè esamineremo le caratteristiche principali dell'instabilità di 
\textit{diocotron}, che è causata da una velocità relativa non nulla presente fra diversi strati di fluido. Il plasma che 
vogliamo trattare è composto da soli elettroni ed a bassa densità: il regime con cui vogliamo lavorare è
\begin{equation}
    \begin{cases}
        \omega_p^2/\Omega^2\, \ll\,1,    \\
        n_i\left(r\right)\,=\,0.
    \end{cases}
\end{equation}
Il plasma elettronico che prendiamo in considerazione è confinato radialmente da un campo magnetico assiale $\vec{B}\,=\,B_0\hat{e}_z$: 
dato il limite in cui ci siamo posti trascuriamo gli effetti inerziali elettronici ($m\,\rightarrow\,0$). L'equazione dei momenti 
fornisce un criterio per determinare il movimento di un elemento di fluido, in quanto il primo membro dipendente dalla massa elettronica 
risulta essere identicamente nullo come evidenziato in seguito
\begin{equation}
    0\,=\,-en\left(\vec{x},\,t\right)\left\{\vec{E}\left(\vec{x},\,t\right)\,+\,\frac{V\left(\vec{x},\,t\right) \times B_0 \hat{e}_z}{c}\left[c\right]\right\}
    \label{equation: mom_eq_dioc}
\end{equation}
Ricavare dalla relazione \eqref{equation: mom_eq_dioc} la velocità media con cui evolve il plasma è immediato: in seguito è riportato il 
risultato e le varie componenti espresse in coordinate polari, che si addicono maggiormente a trattare le simmetrie di nostro interesse. 
\begin{equation}
    \vec{V}\left(\vec{x},\,t\right)\,=\,-\frac{c}{B_0}\nabla \Phi \left(\vec{x},\,t\right) \times \hat{e}_z\left[\frac{1}{c}\right]
    \label{equation: velocità_noinerzia}
\end{equation}
\begin{equation}
    V_r\left(r,\,\theta,\,t\right)\,=\,-\frac{c}{B_0 r}\left[\frac{1}{c}\right]\frac{\partial}{\partial \theta} \left[\Phi\left(r,\,\theta,\,t\right)\right]
\end{equation}
\begin{equation}
    V_\theta\left(r,\,\theta,\,t\right)\,=\,\frac{c}{B_0}\left[\frac{1}{c}\right]\frac{\partial}{\partial r}\left[\Phi\left(r,\,\theta,\,t\right)\right]
\end{equation}
Notiamo che la velocità calcolata con la relazione \eqref{equation: velocità_noinerzia} ha divergenza nulla: questo comporta la scomparsa di 
uno dei tre termini che costituiscono l'equazione di continuità, che insieme all'equazione di Poisson fornisce tutte le informazioni necessarie 
per descrivere l'evoluzione del sistema. Espresse in coordinate polari, esse risultano:
\begin{equation}
    \left\{\frac{\partial}{\partial t}\,-\,\frac{c}{B_0 r}\left[\frac{1}{c}\right]\frac{\partial \Phi}{\partial \theta}\frac{\partial}{\partial r}\,+\,\frac{c}{B_0 r}\left[\frac{1}{c}\right]\frac{\partial \Phi}{\partial r}\frac{\partial}{\partial \theta}\right\} n\left(r,\,\theta,\,t\right)\,=\,0
    \label{equation: continuità_no_inerziapol}
\end{equation}
\begin{equation}
    \left(\frac{1}{r}\frac{\partial}{\partial r}r\frac{\partial}{\partial r}\,+\,\frac{1}{r^2}\frac{\partial^2}{\partial \theta^2}\right)\Phi\left(r,\,\theta,\,t\right)\,=\,4\pi \left[\frac{1}{4\pi\varepsilon_0}\right]e n\left(r,\,\theta,\,t\right)
    \label{equation: poisson_no_inerziapol}
\end{equation}
Notiamo che nel limite di $m\,\rightarrow\,0$ il rapporto $\left(\omega_p/\Omega\right)^2$ che definisce le condizioni in cui stiamo considerando 
il plasma si avvicina allo zero, mentre la frequenza di diocotron
\begin{equation}
    \omega_D\left(r\right)\,=\,\frac{\omega_p^2}{2\Omega}\,=\,\frac{2\pi n_0\left(r\right)ec}{B_0}
    \label{equation: diocotron_freq}
\end{equation}
rimane finita.

\section{Teoria perturbativa lineare}

