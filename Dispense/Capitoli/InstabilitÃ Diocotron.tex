\chapter{Instabilità di diocotron}

Nel capitolo precedente, la velocità angolare era indipendente dalla posizione radiale (ossia $\partial \omega/\partial r\,=\,0$): 
in questo capitolo non è più presente questa restrizione poichè esamineremo le caratteristiche principali dell'instabilità di 
\textit{diocotron}, che è causata da una velocità relativa non nulla presente fra diversi strati di fluido. Il plasma che 
vogliamo trattare è composto da soli elettroni ed a bassa densità: il regime con cui vogliamo lavorare è
\begin{equation}
    \begin{cases}
        \omega_p^2/\Omega^2\, \ll\,1,    \\
        n_i\left(r\right)\,=\,0.
    \end{cases}
\end{equation}
Il plasma elettronico che prendiamo in considerazione è confinato radialmente da un campo magnetico assiale $\vec{B}\,=\,B_0\hat{e}_z$: 
dato il limite in cui ci siamo posti trascuriamo gli effetti inerziali elettronici ($m\,\rightarrow\,0$). L'equazione dei momenti 
fornisce un criterio per determinare il movimento di un elemento di fluido, in quanto il primo membro dipendente dalla massa elettronica 
risulta essere identicamente nullo come evidenziato in seguito
\begin{equation}
    0\,=\,-en\left(\vec{x},\,t\right)\left\{\vec{E}\left(\vec{x},\,t\right)\,+\,\frac{V\left(\vec{x},\,t\right) \times B_0 \hat{e}_z}{c}\left[c\right]\right\}
    \label{equation: mom_eq_dioc}
\end{equation}
Ricavare dalla relazione \eqref{equation: mom_eq_dioc} la velocità media con cui evolve il plasma è immediato: in seguito è riportato il 
risultato e le varie componenti espresse in coordinate polari, che si addicono maggiormente a trattare le simmetrie di nostro interesse. 
\begin{equation}
    \vec{V}\left(\vec{x},\,t\right)\,=\,-\frac{c}{B_0}\nabla \Phi \left(\vec{x},\,t\right) \times \hat{e}_z\left[\frac{1}{c}\right]
    \label{equation: velocità_noinerzia}
\end{equation}
\begin{equation}
    V_r\left(r,\,\theta,\,t\right)\,=\,-\frac{c}{B_0 r}\left[\frac{1}{c}\right]\frac{\partial}{\partial \theta} \left[\Phi\left(r,\,\theta,\,t\right)\right]
\end{equation}
\begin{equation}
    V_\theta\left(r,\,\theta,\,t\right)\,=\,\frac{c}{B_0}\left[\frac{1}{c}\right]\frac{\partial}{\partial r}\left[\Phi\left(r,\,\theta,\,t\right)\right]
\end{equation}
Notiamo che la velocità calcolata con la relazione \eqref{equation: velocità_noinerzia} ha divergenza nulla: questo comporta la scomparsa di 
uno dei tre termini che costituiscono l'equazione di continuità, che insieme all'equazione di Poisson fornisce tutte le informazioni necessarie 
per descrivere l'evoluzione del sistema. Espresse in coordinate polari, esse risultano:
\begin{equation}
    \left\{\frac{\partial}{\partial t}\,-\,\frac{c}{B_0 r}\left[\frac{1}{c}\right]\frac{\partial \Phi}{\partial \theta}\frac{\partial}{\partial r}\,+\,\frac{c}{B_0 r}\left[\frac{1}{c}\right]\frac{\partial \Phi}{\partial r}\frac{\partial}{\partial \theta}\right\} n\left(r,\,\theta,\,t\right)\,=\,0
    \label{equation: continuità_no_inerziapol}
\end{equation}
\begin{equation}
    \left(\frac{1}{r}\frac{\partial}{\partial r}r\frac{\partial}{\partial r}\,+\,\frac{1}{r^2}\frac{\partial^2}{\partial \theta^2}\right)\Phi\left(r,\,\theta,\,t\right)\,=\,4\pi \left[\frac{1}{4\pi\varepsilon_0}\right]e n\left(r,\,\theta,\,t\right)
    \label{equation: poisson_no_inerziapol}
\end{equation}
Notiamo che nel limite di $m\,\rightarrow\,0$ il rapporto $\left(\omega_p/\Omega\right)^2$ che definisce le condizioni in cui stiamo considerando 
il plasma si avvicina allo zero, mentre la frequenza di diocotron
\begin{equation}
    \omega_D\left(r\right)\,=\,\frac{\omega_p^2}{2\Omega}\,=\,\frac{2\pi n_0\left(r\right)ec}{B_0}
    \label{equation: diocotron_freq}
\end{equation}
rimane finita.

\section{Teoria perturbativa lineare}

Per investigare le proprietà del set di equazioni introdotto nella sezione precedente, lavoriamo con piccole perturbazioni che 
rompano la simmetria azimutale: supponiamo di introdurre delle eccitazioni che presentano una dipendenza dalla coordinata angolare 
$\theta$ in modo tale da avere una densità ed un potenziale elettrostatico pari a 
\begin{align}
    & n\left(r,\,\theta,\,t\right)\,=\,n_0\left(r\right)\,+\,n_1\left(r,\,\theta,\,t\right)\,=\,n_0\left(r\right)\,+\,\sum_{l\,=\,-\infty}^{+\infty} n_1^l\left(r\right)\exp{\left(il\theta\,-\,i\omega_l t\right)}, \\
    & \Phi\left(r,\,\theta,\,t\right)\,=\,\Phi_0\left(r\right)\,+\,\Phi_1\left(r,\,\theta,\,t\right)\,=\,\Phi_0\left(r\right)\,+\,\sum_{l\,=\,-\infty}^{+\infty} \Phi_1^l\left(r\right)\exp{\left(il\theta\,-\,i\omega_l t\right)},
\end{align}
dove il parametro $l$ è il numero che descrive il modo azimutale della perturbazione. Andando a sostituire la densità $n\left(r,\,\theta,\,t\right)$ 
nell'equazione di continuità possiamo compiere il primo passo per la determinazione delle autofrequenze dei modi di diocotron e delle autofunzioni: dato 
che vogliamo effettuare un'analisi lineare trascuriamo tutti i termini di ordine superiore nella perturbazione. Considerando i singoli 
termini abbiamo che
\begin{align}
    & \frac{\partial \Phi}{\partial \theta}\frac{\partial n}{\partial r}\,=\,-i\frac{\partial}{\partial r}\left[n_0\left(r\right)\right]\sum_{l\,=\,-\infty}^{+\infty}l\Phi_1^l\left(r\right)\exp{\left(il\theta\,-\,i\omega_l t\right)}\\
    & \frac{\partial \Phi}{\partial r}\frac{\partial n}{\partial \theta}\,=\,i\frac{\partial}{\partial r}\left[\Phi_0\left(r\right)\right]\sum_{l\,=\,-\infty}^{+\infty}ln_1^l\left(r\right)\exp{\left(il\theta\,-\,i\omega_l t\right)}\\
    & \frac{\partial n}{\partial t}\,=\,-i\sum_{l\,=\,-\infty}^{+\infty} \omega_l n_1^l\left(r\right)\exp{\left(il\theta\,-\,i\omega_l t\right)}
\end{align}
Introducendo quanto trovato nell'equazione di continuità e considerando l'evoluzione di uno specifico modo di diocotron abbiamo che
\begin{equation}
    n_1^l\left(r\right)\left\{\frac{cl}{B_0 r}\left[\frac{1}{c}\right]\frac{\partial}{\partial r}\left[\Phi_0\left(r\right)\right]\,-\,\omega_l\right\}\,=\,\frac{cl}{B_0r}\left[\frac{1}{c}\right]\Phi_1^l\left(r\right)\frac{\partial}{\partial r}\left[n_0\left(r\right)\right] 
    \label{equation: continuity_diocotronmode}
\end{equation}
Effettuiamo la stessa analisi con l'equazione di Poisson: vogliamo introdurre le perturbazioni non-assisimmetriche in un contesto di 
analisi perturbativa lineare. Consideriamo anche in questo caso i singoli termini, ottenendo che 
\begin{align}
    & \frac{\partial^2}{\partial \theta^2}\left[\Phi\left(r,\,\theta,\,t\right)\right]\,=\,-\sum_{l\,=\,-\infty}^{+\infty} l^2 \Phi_1^l\left(r\right)\exp{\left(il\theta\,-\,i\omega_l t\right)}\\
    & \frac{1}{r}\frac{\partial}{\partial r}\left\{r\frac{\partial}{\partial r}\left[\Phi\left(r,\,\theta,\,t\right)\right]\right\}\,=\,\frac{1}{r}A\left(r,\,\theta,\,t\right)\,+\,B\left(r,\,\theta,\,t\right)
\end{align}
dove
\begin{align}
    & A\left(r,\,\theta,\,t\right)\,=\,\frac{\partial}{\partial r}\left[\Phi_0\left(r\right)\right]\,+\,\sum_{l\,=\,-\infty}^{+\infty}\frac{\partial}{\partial r}\left[\Phi_1^l\left(r\right)\right]\exp{\left(il\theta\,-\,i\omega_l t\right)} \\
    & B\left(r,\,\theta,\,t\right)\,=\,\frac{\partial^2}{\partial r^2}\left[\Phi_0\left(r\right)\right]\,+\,\sum_{l\,=\,-\infty}^{+\infty}\frac{\partial ^2}{\partial r^2}\left[\Phi_1^2\left(r\right)\right]\exp{\left(il\theta\,-\,i\omega_l t\right)}
\end{align}
Notiamo ora che possiamo distinguere due equazioni, una per i termini di grado zero che è l'equazione di Poisson che abbiamo già
utilizzato in precedenza ed una seconda per i termini di grado uno, infatti si ha che:
\begin{equation}
    \frac{1}{r}\frac{\partial}{\partial r}\left[\Phi_0\left(r\right)\right]\,+\,\frac{\partial^2}{\partial r^2}\left[\Phi_0\left(r\right)\right]\,=\,4\pi \left[\frac{1}{4\pi\varepsilon_0}\right] en_0\left(r\right)
    \label{equation: Poisson_termini0}
\end{equation}
\begin{equation}
    \frac{1}{r}\frac{\partial}{\partial r}\left\{r\frac{\partial}{\partial r}\left[\Phi_1^l\left(r,\,\theta,\,t\right)\right]\right\}\,-\,l^2\Phi_1^l\left(r\right)\,=\,4\pi \left[\frac{1}{4\pi\varepsilon_0}\right] en_1^l\left(r\right)
    \label{equation: Poisson_termini1}
\end{equation}
Possiamo riscrivere il risultato appena ottenuto per le perturbazioni di primo ordine al potenziale riconoscendo la presenza in \eqref{equation: continuity_diocotronmode} 
di frequenze caratteristiche dei fenomeni di plasma quali:
\begin{align}
    & \Omega\,=\,\frac{eB_0}{mc} \qquad \qquad \qquad \qquad \qquad \,\,\,\,\,\,\,\, \text{frequenza di ciclotrone}\\
    & \omega_p^2\left(r\right)\,=\,4\pi\left[\frac{1}{4\pi\varepsilon_0}\right]\frac{n_0\left(r\right)}{m} \, \qquad \qquad \text{frequenza di plasma} \\
    & \omega^-\left(r\right)\,=\,\frac{c}{rB_0}\left[\frac{1}{c}\right]\frac{\partial}{\partial r}\left[\Phi_0\left(r\right)\right] \qquad \,\,\,\, \text{frequenza di rotazione}
\end{align}
e sostituendo $n_1\left(r\right)$ nell'equazione di Poisson per i termini di primo ordine.
Così facendo troviamo il problema agli autovalori, che può essere utilizzato per determinare le autofrequenze $\omega_l$ e di conseguenza 
le autofunzioni $\Phi_l\left(r\right)$:
\begin{equation}
    \frac{1}{r}\frac{\partial}{\partial r}\left\{r\frac{\partial}{\partial r}\left[\Phi^l_1\left(r\right)\right]\right\}\,-\,\left(\frac{l}{r}\right)^2\Phi_1^l\left(r\right)\,=\,-\frac{l}{\Omega r}\frac{\partial}{\partial r}\left[\omega_p^2\left(r\right)\right]\frac{\Phi_1^l\left(r\right)}{\omega_l\,-\,l\omega^-\left(r\right)}
    \label{equation: problema_autovalori}
\end{equation}

\subsection{Profilo di densità a gradino}

Supponiamo che il profilo di densità $\rho\left(r,\,\theta\right)\,=\,\rho\left(\theta\right)$ sia quello a gradino introdotto
in precedenza. Il salto avviene per la posizione radiale $r\,=\,R_p$ ed il raggio della regione confinante della trappola 
è $r\,=\,R_w$:
\begin{equation}
    n_0\left(r\right)\,=\,
    \begin{cases}
        n_0     \qquad        r \leq R_p \\
        0       \,\,\, \qquad        R_p < r \leq R_w 
    \end{cases}
\end{equation}
Vogliamo trovare le autofrequenze ed i modi normali per un sistema siffatto: iniziamo lavorando con l'equazione di Poisson che abbiamo
determinato in precedenza. Raccogliamo la frequenza di ciclotrone in modo tale da ottenere che
\begin{equation}
    \frac{\partial}{\partial r}\left\{r\frac{\partial}{\partial r}\left[\Phi^l_1\left(r\right)\right]\right\}\,-\,\frac{l^2}{r}\Phi_1^l\left(r\right)\,=\,-2l \frac{\partial}{\partial r}\left[\frac{\omega_p^2\left(r\right)}{2\Omega}\right]\frac{\Phi_1^l\left(r\right)}{\omega_l\,-\,l\omega^-\left(r\right)}
    \label{equation: diffPhi_rhoGrad}
\end{equation}
La derivata del rapporto fra la frequenza di plasma elevata al quadarto ed il doppio della frequenza di ciclotrone è diversa da zero 
solamente nella discontinuita posta ad $r\,=\,R_p$. Detto $\omega_D$, oppure frequenza angolare di Diocotron, il rapporto precedentemente
citato, si deve avere che:
\begin{equation}
    \frac{d \omega_D}{dr}\,=\,-\omega_D\delta\left(r\,-\,R_p\right)
    \label{equation: omegaD_discontinuità}
\end{equation}
Quando la posizione radiale con coincide con il raggio di plasma, il secondo membro risulta essere identicamente nullo. Le soluzioni da
ricercare per una equazione differenziale di questo genere sono del tipo $r^{\lambda}$: sostituendo un tale ansaltz nell'equazione
\begin{equation}
    \frac{\partial}{\partial r}\left\{r\frac{\partial}{\partial r}\left[\Phi^l_1\left(r\right)\right]\right\}\,-\,\frac{l^2}{r}\Phi_1^l\left(r\right)\,=\,0
    \label{equation:autoF_noRp}
\end{equation}
si può ricavare come $\lambda\,=\,\pm l$. Le condizioni che consentono di univocamente determinare la soluzione sono la non-divergenza 
della stessa quando il raggio è identicamente nullo, l'annullamento della perturbazione di potenziale sulla parete della trappola ed 
il collegamento con continuita della soluzione fra le due regioni prese in considerazione. Imponendo le prime due ci riportiamo ad un
andamento del tipo
\begin{equation}
    \Phi_1^l\left(r\right)\,=\,
    \begin{cases}
        A\left(r/R_w\right)^l \qquad \qquad \qquad \qquad \qquad r\,\leq\,R_p\\
        C\left[\left(r/Rw\right)^l\,-\,\left(r/R_w\right)^{-l}\right] \qquad R_p\,<\,r\,\leq\,R_w
    \end{cases}
    \label{equation: sol_modo1Dioc_passo1}
\end{equation}
Consideriamo ora quanto accade all'interfaccia fra le due regioni dove abbiamo calcolato le soluzioni. Richiediamo innanzitutto che
il potenziale sia continuo al'interfaccia; questo si traduce in
\begin{equation}
    A\left(\frac{R_p}{R_w}\right)^l\,=\,C\left[\left(\frac{R_p}{R_w}\right)^l\,-\,\left(\frac{R_p}{R_w}\right)^{-l}\right]
\end{equation}
Il campo presenta tuttavia una discontinuità: questo è dovuto alla presenza della delta in corrispondenza di $r\,=\,R_p$. Quantifichiamo 
quale sia il contributo della delta nella regione di cambio di densità integrando in un intervallo $\varepsilon$ piccolo a piacere la
relazione \eqref{equation: diffPhi_rhoGrad}. Abbiamo quindi che 
\begin{equation}
    \int_{R_p-\varepsilon}^{R_P+\varepsilon} \left[    \frac{\partial}{\partial r}\left\{r\frac{\partial}{\partial r}\left[\Phi^l_1\left(r\right)\right]\right\}\,-\,\frac{l^2}{r}\Phi_1^l\left(r\right)\right] dr\,=\,-\int_{R_p-\varepsilon}^{R_p+\varepsilon} \left[2l\frac{\Phi_1^l}{\omega_l\,-l\omega_D}\frac{d\omega_D}{dr}\right] dr
    \label{equation: qunatif_discE}
\end{equation}
Le equazioni \eqref{equation: sol_modo1Dioc_passo1} ed \eqref{equation: qunatif_discE} costituiscono un sistema lineare di rapida 
risoluzione. Le autofrequenze e le autofunzioni in analisi lineare per il semplice profilo di densità preso in considerazione 
risultano essere
\begin{equation}
    \omega_l\,=\,\omega_D\left[l\,-\,1\,+\,\left(\frac{R_p}{R_w}\right)^2\right]
    \label{equation: autoFreq_lineare}
\end{equation}
\begin{equation}
    \Phi^l\,=\,
    \begin{cases}
        A\left(r/R_w\right)^l  \qquad \qquad \qquad \qquad \qquad \quad r\,\leq\,R_p \\
        A\left(r/R_w\right)^l \frac{\left(r/R_w\right)^l\,-\,\left(r/R_w\right)^{-l}}{\left(R_p/R_w\right)^l\,-\,\left(R_p/R_w\right)^{-l}} \qquad  R_p\,<\,r\,\leq\,R_w
    \end{cases}
    \label{equation: autoFunc_lineare}
\end{equation}
